% Documento LaTeX com o artigo que estamos escrevendo

% Cabeçalho
% configura o documento 
%%%%%%%%%%%%%%%
\documentclass{article}

\usepackage [brazil]{babel}
\usepackage{graphicx} 
\usepackage [round,authoryear,sort]{natbib} %formatação de referências 

\newcommand{\Title}{Análise de variação de temperatura dos últimos cinco anos} % variáveis
\input{paises.tex}


% Corpo
% onde escreve o texto 
%%%%%%%%%%%%%%%%%
\begin{document}

\title{\Title} % outra forma de escrever 
\author{Julia}

\maketitle 
% \pagebreak para quebra de página 

\begin{abstract}
Meu resumo 
\end{abstract}
	Meu artigo. 
	
\section{Introdução}
Isso vai ser a introdução.
Outra frase.

Outro parágrafo. %pulando linha 

Trabalhos anteriores bem legais fizeram coisas parecidas 
\citep{}.	% coloca o códigozinho 
Isso foi analisado primeiro por \citet{}. % mesmo código 

\section{Metodologia}
\label{sec:metodos}

Aqui eu vou descrever tudo que eu fiz. Ajustamos uma reta aos cinco último anos dos dados 
de temperatura média mensal para cada país. 

Assim, calculamos a taxa de variação da temperatura recente. 


A equação da reta é 

\begin{equation}
T(t) = a t + b, 
\label{eq:reta}
\end{equation}

\noindent
onde $T$ é a temperatura, $t$ é o tempo, $a$ é o coeficiente angular e $b$ é o coeficiente linear. 

Utilizamos a equação \ref{eq:reta} em um código python para fazer um ajuste da reta com o método dos mínimos quadrados. Na seção \ref{sec:metodos}.

\section{Resultados}

Analisamos os dados de 225 países. São muitos dados. 
Os países analisados foram: \Paises. 

\begin{figure}[!htb] %para redirecionar a figura 
\centering 
\includegraphics{../figuras/variacao_temperatura.png} %[width=\columnwidth] 
\caption{Variação de temperatura média mensal dos cinco últimos anos. a) Países com as cinco maiores variações de temperaturas. 
b) Países com as cinco menores variações de temperaturas.}
\label{fig:variacao}
\end{figure}

Os resultados da análise de variação de temperatura estão na figura \ref{fig:variacao}.

\bibliographystyle{apalike}
\bibliography{} % arquivo salvo com a referência 

\end{document} 